% Homework template for Learning from Data
% by Xiangxiang Xu <xiangxiangxu.thu@gmail.com>
% LAST UPDATE: October 8, 2018
\documentclass[a4paper]{article}
\usepackage[T1]{fontenc}
\usepackage{amsmath, amssymb, amsthm}
% amsmath: equation*, amssymb: mathbb, amsthm: proof
\usepackage{moreenum}
\usepackage{mathtools}
\usepackage{url}
\usepackage{enumitem}
\usepackage{bm}
\usepackage{graphicx}
\usepackage{subcaption}
\usepackage{booktabs} % toprule
\usepackage[mathcal]{eucal}
\usepackage{dsfont}
\usepackage[numbered,framed]{matlab-prettifier}
%% Definitions for Learning from Data
%% UPDATE: October 8, 2018 by Xiangxiang 
\newcommand{\theterm}{Fall 2018}

\newcommand{\thecoursename}{
Tsinghua-Berkeley Shenzhen Institute\\
%\vspace*{0.1in}
\textsc{Learning from Data}
}

\newcommand{\courseheader}{
\vspace*{-1in}
\begin{center}
\thecoursename \\
\theterm
\vspace*{0.1in}
\hrule
\end{center}
}
\newcommand{\uc}{\underline{c}}    % c, vec
\newcommand{\uv}{\underline{v}}    % x, vec
\newcommand{\uw}{\underline{w}}    % w, vec
\newcommand{\ux}{\underline{x}}    % x, vec
\newcommand{\uy}{\underline{y}}    % y, vec
\newcommand{\uz}{\underline{z}}    % z, vec
\newcommand{\um}{\underline{m}}    % m, vec
\newcommand{\ut}{\underline{t}}    % t, vec
\newcommand{\bc}{\bm{c}}    % c, vec
\newcommand{\bv}{\bm{v}}    % x, vec
\newcommand{\bw}{\bm{w}}    % w, vec
\newcommand{\bx}{\bm{x}}    % x, vec
\newcommand{\by}{\bm{y}}    % y, vec
\newcommand{\bz}{\bm{z}}    % z, vec
% \newcommand{\bm}{\bm{m}}    % m, vec
\newcommand{\bt}{\bm{t}}    % t, vec

\newcommand{\balpha}{\bm{\alpha}}    % alpha, vec
\newcommand{\bxi}{\bm{\xi}}    % xi, vec


\newcommand{\rvx}{\mathsf{x}}    % x, r.v.
\newcommand{\rvy}{\mathsf{y}}    % y, r.v.
\newcommand{\rvz}{\mathsf{z}}    % z, r.v.
\newcommand{\rvw}{\mathsf{w}}    % w, r.v.
\newcommand{\rvv}{\mathsf{v}}    % v, r.v.
\newcommand{\rvm}{\mathsf{m}}    % m, r.v.
\newcommand{\rvt}{\mathsf{t}}    % t, r.v.
\newcommand{\rvH}{\mathsf{H}}    % H, r.v.
\newcommand{\urvx}{\underline{\mathsf{x}}}    % x, r.v. vec
\newcommand{\urvy}{\underline{\mathsf{y}}}    % y, r.v. vec
\newcommand{\urvz}{\underline{\mathsf{z}}}    % z, r.v. vec
\newcommand{\urvw}{\underline{\mathsf{w}}}    % w, r.v. vec
\newcommand{\urvt}{\underline{\mathsf{t}}}    % t, r.v. vec
\newcommand{\defeq}{\triangleq} %\coloneqq
\newcommand{\reals}{\mathbb{R}}
\newcommand{\T}{\mathrm{T}}    % transpose
\newcommand{\BLS}{\mathrm{BLS}}    % BLS
\newcommand{\LLS}{\mathrm{LLS}}    % LLS
\newcommand{\MVU}{\mathrm{MVU}}    % MVU
\newcommand{\vecf}{\bm{\mathrm{vec}}}   % vec()

\DeclareMathOperator*{\maximize}{maximize}    % maximize
\DeclareMathOperator*{\minimize}{minimize}    % minimize
\newcommand{\st}{\mathrm{subject~to}}    % minimize


% \newcommand{\E}[1]{\mathbb{E}\left[{#1}\right]}
% \newcommand{\Prob}[1]{\mathbb{P}\left({#1}\right)}
\DeclareMathOperator*{\argmax}{arg\,max}
\DeclareMathOperator*{\argmin}{arg\,min}
\DeclareMathOperator*{\argsup}{arg\,sup}
\DeclareMathOperator*{\arginf}{arg\,inf}
\DeclareMathOperator{\Var}{Var}
\DeclareMathOperator{\Cov}{Cov}
\DeclareMathOperator{\MSE}{MSE}
\DeclareMathOperator{\1}{\mathds{1}}
\DeclareMathOperator{\E}{\mathbb{E}}
\DeclareMathOperator{\Prob}{\mathbb{P}}

\newcommand\independent{\protect\mathpalette{\protect\independenT}{\perp}}
\def\independenT#1#2{\mathrel{\rlap{$#1#2$}\mkern2mu{#1#2}}}


\lstset{
  style              = Matlab-editor,
  captionpos         =b,
  basicstyle         = \mlttfamily,
  escapechar         = ",
  mlshowsectionrules = true,
}

\begin{document}
\courseheader

\newcounter{hwcnt}
\setcounter{hwcnt}{3}

\begin{center}
    \underline{\bf Homework \thehwcnt} \\
\end{center}
\begin{flushleft}
    Zi-Feng WANG\hfill
    \today
\end{flushleft}
\hrule
\vspace{2em}
\setlist[enumerate,1]{label=\thehwcnt.\arabic*.}
\setlist[enumerate,2]{label={(\alph*)}
\setlist[enumerate,3]{label=\roman*.}
\setlist[enumerate,4]{label=\greek*)}

\flushleft
\rule{\textwidth}{1pt}
\begin{itemize}
\item {\bf Acknowledgments: \/}
    This template takes some materials from course CSE 547/Stat 548 of Washington University: \small 
    {\url{https://courses.cs.washington.edu/courses/cse547/17sp/index.html}}.
\item {\bf Collaborators: \/}
    I finish my homework all by myself.
\end{itemize}
\rule{\textwidth}{1pt}

\vspace{2em}


%%% Now Start the main part of this assignment. %%%

\begin{enumerate}
\item \textbf{Solution:}
when $K=2$, the conditional distribution of data $x$ is
\begin{equation}
Pr(x^{i}|y^i=j;\theta) = \rm{Norm}_x(\mu_j,\Sigma_j), \quad j \in \{1,\cdots,K\}
\end{equation}
the log likelihood function is
\begin{equation}
\begin{align}
\mathcal{L} &= \log \prod_{i=1}^m Pr(x^i|y^i;\theta) Pr(y^i|\theta) \\
&= \sum_{i=1}^m \log Pr(x^i|y^i;\theta) + \log Pr(y^i|\theta) 
\end{align}
\end{equation}
the derivative of $\mathcal{L}$ by $\Sigma_j$ is
\begin{equation}
    \frac{\partial\mathcal{L}}{\partial\Sigma_j} = 
    \sum_{i=1}^m \frac1{Pr(x^i|y^i;\theta)}\frac{\partial Pr(x^i|y^i;\theta)}{\partial \Sigma_j} \1\{y^i=j\}
    \label{eq:3}
\end{equation}
as 
\begin{equation}
\begin{align}
    \frac{\partial Pr(x^i|y^i;\theta)}{\partial \Sigma_j} &=
    \frac1{(2\pi)^{n/2}}e^{-0.5(x-\mu_j)^{\T}\Sigma_j^{-1}(x-\mu_j)} \\
    &* \{\frac{\partial\frac1{|\Sigma_j|^{1/2}}}{\partial\Sigma_j} 
    + \frac1{|\Sigma_j|^{1/2}} \frac{\partial(-0.5(x-\mu_j)^{\T}\Sigma_j^{-1}(x-\mu_j))}{\partial\Sigma_j} \}\\
    &= \frac1{(2\pi)^{n/2}}e^{-0.5(x-\mu_j)^{\T}\Sigma_j^{-1}(x-\mu_j)} \\
    &* \{-\frac12|\Sigma_j|^{-1/2}\Sigma_j^{-1} + \frac12|\Sigma_j|^{-1/2}\Sigma_j^{-1}
    (x-\mu_j)(x-\mu_j)^{\T}\Sigma_j^{-1}\} \\
    &= \frac1{(2\pi)^{n/2}}e^{-0.5(x-\mu_j)^{\T}\Sigma_j^{-1}(x-\mu_j)} \\
    &* -\frac12|\Sigma_j|^{-1/2}\Sigma_j^{-1} \{I - (x-\mu_j)(x-\mu_j)^{\T}\Sigma_j^{-1}\}
\end{align}
\label{eq:4}
\end{equation}
combine the \eqref{eq:3} and \eqref{eq:4}, then let $\frac{\partial\mathcal{L}}{\partial\Sigma_j} = 0$, we get
\begin{equation}
\begin{aligned}
 -\frac12\sum_{i=1}^m\1\{ y^i=j\}\Sigma_j^{-1}[I-(x-\mu_j)(x-\mu_j)^{\T}\Sigma_j^{-1}] = 0 \\
 \Sigma_j \sum_{i=1}^m\1\{y^i=j\} = \sum_{i=1}^m\1\{y^i=j\}(x-\mu_j)(x-\mu_j)^{\T} 
\end{aligned}
\end{equation}
therefore the maximum likelihood estimate of $\Sigma_j$ for the case of $K=2$ in QDA model is
\begin{equation}
\hat{\Sigma_j} = \frac{ \sum_{i=1}^m\1\{y^i=j\}(x^i-\mu_j)(x^i-\mu_j)^{\T} }{\sum_{i=1}^m \1\{y^i=j\}}
\end{equation}
\end{enumerate}


\end{document}
