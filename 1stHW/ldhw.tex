% Homework template for Learning from Data
% by Xiangxiang Xu <xiangxiangxu.thu@gmail.com>
% LAST UPDATE: October 8, 2018
\documentclass[a4paper]{article}
\usepackage[T1]{fontenc}
\usepackage{amsmath, amssymb, amsthm}
% amsmath: equation*, amssymb: mathbb, amsthm: proof
\usepackage{moreenum}
\usepackage{mathtools}
\usepackage{url}
\usepackage{enumitem}
\usepackage{bm}
\usepackage{graphicx}
\usepackage{subcaption}
\usepackage{booktabs} % toprule
\usepackage[mathcal]{eucal}
\usepackage{dsfont}
\usepackage[numbered,framed]{matlab-prettifier}
%% Definitions for Learning from Data
%% UPDATE: October 8, 2018 by Xiangxiang 
\newcommand{\theterm}{Fall 2018}

\newcommand{\thecoursename}{
Tsinghua-Berkeley Shenzhen Institute\\
%\vspace*{0.1in}
\textsc{Learning from Data}
}

\newcommand{\courseheader}{
\vspace*{-1in}
\begin{center}
\thecoursename \\
\theterm
\vspace*{0.1in}
\hrule
\end{center}
}
\newcommand{\uc}{\underline{c}}    % c, vec
\newcommand{\uv}{\underline{v}}    % x, vec
\newcommand{\uw}{\underline{w}}    % w, vec
\newcommand{\ux}{\underline{x}}    % x, vec
\newcommand{\uy}{\underline{y}}    % y, vec
\newcommand{\uz}{\underline{z}}    % z, vec
\newcommand{\um}{\underline{m}}    % m, vec
\newcommand{\ut}{\underline{t}}    % t, vec
\newcommand{\bc}{\bm{c}}    % c, vec
\newcommand{\bv}{\bm{v}}    % x, vec
\newcommand{\bw}{\bm{w}}    % w, vec
\newcommand{\bx}{\bm{x}}    % x, vec
\newcommand{\by}{\bm{y}}    % y, vec
\newcommand{\bz}{\bm{z}}    % z, vec
% \newcommand{\bm}{\bm{m}}    % m, vec
\newcommand{\bt}{\bm{t}}    % t, vec

\newcommand{\balpha}{\bm{\alpha}}    % alpha, vec
\newcommand{\bxi}{\bm{\xi}}    % xi, vec


\newcommand{\rvx}{\mathsf{x}}    % x, r.v.
\newcommand{\rvy}{\mathsf{y}}    % y, r.v.
\newcommand{\rvz}{\mathsf{z}}    % z, r.v.
\newcommand{\rvw}{\mathsf{w}}    % w, r.v.
\newcommand{\rvv}{\mathsf{v}}    % v, r.v.
\newcommand{\rvm}{\mathsf{m}}    % m, r.v.
\newcommand{\rvt}{\mathsf{t}}    % t, r.v.
\newcommand{\rvH}{\mathsf{H}}    % H, r.v.
\newcommand{\urvx}{\underline{\mathsf{x}}}    % x, r.v. vec
\newcommand{\urvy}{\underline{\mathsf{y}}}    % y, r.v. vec
\newcommand{\urvz}{\underline{\mathsf{z}}}    % z, r.v. vec
\newcommand{\urvw}{\underline{\mathsf{w}}}    % w, r.v. vec
\newcommand{\urvt}{\underline{\mathsf{t}}}    % t, r.v. vec
\newcommand{\defeq}{\triangleq} %\coloneqq
\newcommand{\reals}{\mathbb{R}}
\newcommand{\T}{\mathrm{T}}    % transpose
\newcommand{\BLS}{\mathrm{BLS}}    % BLS
\newcommand{\LLS}{\mathrm{LLS}}    % LLS
\newcommand{\MVU}{\mathrm{MVU}}    % MVU
\newcommand{\vecf}{\bm{\mathrm{vec}}}   % vec()

\DeclareMathOperator*{\maximize}{maximize}    % maximize
\DeclareMathOperator*{\minimize}{minimize}    % minimize
\newcommand{\st}{\mathrm{subject~to}}    % minimize


% \newcommand{\E}[1]{\mathbb{E}\left[{#1}\right]}
% \newcommand{\Prob}[1]{\mathbb{P}\left({#1}\right)}
\DeclareMathOperator*{\argmax}{arg\,max}
\DeclareMathOperator*{\argmin}{arg\,min}
\DeclareMathOperator*{\argsup}{arg\,sup}
\DeclareMathOperator*{\arginf}{arg\,inf}
\DeclareMathOperator{\Var}{Var}
\DeclareMathOperator{\Cov}{Cov}
\DeclareMathOperator{\MSE}{MSE}
\DeclareMathOperator{\1}{\mathds{1}}
\DeclareMathOperator{\E}{\mathbb{E}}
\DeclareMathOperator{\Prob}{\mathbb{P}}

\newcommand\independent{\protect\mathpalette{\protect\independenT}{\perp}}
\def\independenT#1#2{\mathrel{\rlap{$#1#2$}\mkern2mu{#1#2}}}


\lstset{
  style              = Matlab-editor,
  captionpos         =b,
  basicstyle         = \mlttfamily,
  escapechar         = ",
  mlshowsectionrules = true,
}

\begin{document}
\courseheader

\newcounter{hwcnt}
\setcounter{hwcnt}{1}

\begin{center}
    \underline{\bf Homework \thehwcnt} \\
\end{center}
\begin{flushleft}
    YOUR NAME\hfill
    \today
\end{flushleft}
\hrule
\vspace{2em}
\setlist[enumerate,1]{label=\thehwcnt.\arabic*.}
\setlist[enumerate,2]{label={(\alph*)}
\setlist[enumerate,3]{label=\roman*.}
\setlist[enumerate,4]{label=\greek*)}

\flushleft
\rule{\textwidth}{1pt}
\begin{itemize}
\item {\bf Acknowledgments: \/}
    This template takes some materials from course CSE 547/Stat 548 of Washington University: \small 
    {\url{https://courses.cs.washington.edu/courses/cse547/17sp/index.html}}.
\item {\bf Collaborators: \/}
    I finish my homework all by myself.
\end{itemize}
\rule{\textwidth}{1pt}

\vspace{2em}


%%% Now Start the main part of this assignment. %%%

Consider the problem of claasifying $l$ samples using SVM, where $ \bx_i \in \mathbb{R}^{n}$, \\ 
$\by_i \in \{-1,1\}$, $(i=1,\dots, l)$.

\begin{enumerate}
    \setlength{\itemsep}{3\parskip}
    \item Suppose the data are linearly separable. The optimization problem of SVM is 
    \begin{equation}
        \tag{P}
        \begin{aligned}
            &\minimize_{\bw,b}& &\frac12 \| \bw\|_2^2\\
            &\st& &y_i(\bw^{\T}\bx_i+b) \geq 1, \quad i = 1, \dots, l,
        \end{aligned}
        \label{eq:svm:p}
    \end{equation}
    and let $(\bw^{\star},b^{\star}) $ denote its optimal solution.
    \begin{enumerate}
    \item Show that
        \begin{equation*}
        b^{\star} = -\frac12 \left(\max_{i\colon y_i=-1} \bw^{\star\T}\bx_i + \min_{i\colon y_i=1}
        \bw^{\star\T}\bx_i \right).
        \end{equation*}

        \noindent The corresponding Lagrange dual problem is given by
        \begin{equation}
            \tag{D}
            \begin{aligned}
            &\maximize_{\balpha}& &\sum_{i=1}^{l} \alpha_i - \frac12 \sum_{i=1}^{l} 
            \sum_{j=1}^{l} \alpha_i \alpha_j y_i y_j \langle \bx_i,\bx_j \rangle\\
            &\st& &\alpha_i \geq 0, \quad i = 1, \dots, l,\\
            &&&\sum_{i=1}^{l} \alpha_i y_i = 0.
            \end{aligned}
            \label{eq:svm:d}
        \end{equation}
        Suppose the optimal solution of \eqref{eq:svm:d} is $\balpha^{\star}=(\alpha_{1}^{\star},
            \cdots \alpha_l^{\star})^{\T}$,from the KKT conditions we know that
        \begin{equation}
        \begin{gathered}
            \bw^{\star} = \sum_{i=1}^{l}\alpha_i^{\star}y_i\bx_i,\\
            \sum_{i=1}^{l}\alpha_i^{\star}\left[y_i(\bw^{\star\T}\bx_i+b^{\star})-1\right] = 0.
        \end{gathered}
        \label{eq:kkt}
        \end{equation}

        \begin{proof}
            From a condition from the \eqref{eq:kkt}:
            \begin{equation*}
            \begin{aligned}
              & \sum_{i=1}^{l}\alpha_i^{\star}\left[y_i(\bw^{\star\T}\bx_i+b^{\star})-1\right] = 0 &\\
              & \sum_{i=1}^{l}\alpha_i^{\star}\left[(\bw^{\star\T}\bx_i+b^{\star})-y_i\right]  = 0 & 
            \end{aligned}
            \end{equation*}
            There is at least one $\alpha_j^{\star} > 0$ 
                that let $\bw^{\star\T}\bx_j + b^{\star} - y_j = 0$.\\
            As the data are linearly separable, for any $y_j=1$, $\bw^{\star}\bx_j + b^{\star} \geq 1$,
            So that \begin{equation*}
                \min_{j:y_j=1}\{\bw^{\star}\bx_j + b^{\star}\} = 1
            \tag{1.a.1}
            \end{equation*}
            Similarily, for any $y_j=-1$, $\bw^{\star}\bx_j + b^{\star} \leq -1$,
            So that \begin{equation*}
                \max_{j:y_j=-1}\{\bw^{\star}\bx_j + b^{\star}\} = -1
            \tag{1.a.2}
            \end{equation*}
            Sum up the equation {1.a.1} and {1.a.2}, here it is:
            \begin{equation*}
                b^{\star} = -\frac12 \left(\max_{i\colon y_i=-1} \bw^{\star\T}\bx_i + \min_{i\colon y_i=1}
                    \bw^{\star\T}\bx_i \right).
            \end{equation*}
        \end{proof}

    \item Based on \eqref{eq:kkt}, verify that
        \begin{equation*}
            \frac{1}{2} \| \bw^{\star} \| _{2}^{2} = \sum_{i=1}^{l}\alpha_i^{\star} - 
            \frac{1}{2} \sum_{i=1}^{l} \sum_{j=1}^{l} \alpha_i^{\star} \alpha_j^{\star}
            y_i y_j \langle \bx_i \bx_j \rangle = \frac12 \sum_{i=1}^{l}\alpha_i^{\star}.
        \end{equation*}
        
       \begin{proof}
            From a condition from the \eqref{eq:kkt}:
            \begin{equation*}
            \begin{aligned}
                 &\sum_{i=1}^{l}\alpha_i^{\star}\left[y_i(\bw^{\star\T}\bx_i+b^{\star})-1\right] = 0 \\
                 &\sum_{i=1}^{l}\alpha_i^{\star}y_i\bw^{\star\T}\bx_i + 
                    b^{\star}\sum_{i=1}^{l}\alpha_i^{\star}y_i - \sum_{i=1}^{l}\alpha_i^{\star} = 0  \\
                 &\sum_{i=1}^{l}\alpha_i^{\star} = \sum_{i=1}^{l}\alpha_i^{\star}y_i\bw^{\star\T}\bx_i 
                    = \bw^{\star\T}\bw^{\star} = \|\bw^{\star}\|_2^2
            \end{aligned}
            \end{equation*}
            And as \begin{equation*}
                \sum_{i=1}^{l}\alpha_i^{\star}y_i\bw^{\star\T}\bx_i = 
                \sum_{i=1}^{l} \sum_{j=1}^{l} \alpha_i^{\star} \alpha_j^{\star} 
                y_i y_j \langle \bx_i,\bx_j \rangle
                \end{equation*}
            Here it is:
            \begin{equation*}
            \begin{aligned}
                \frac12 \|\bw^{\star} \|_2^2 &= \bw^{\star\T}\bw^{\star} - 
                    \frac12 \bw^{\star\T}\bw^{\star}\\
                    &= \sum_{i=1}^{l}\alpha_i^{\star} - 
                    \frac12 \sum_{i=1}^{l}\sum_{j=1}^{l}  \alpha_i^{\star}\alpha_j^{\star} 
                        y_i y_j \langle \bx_i \bx_j \rangle 
            \end{aligned}
            \end{equation*}
        \end{proof}
    \end{enumerate}

    \item When the data are not linearly separable, consider the soft-margin SVM given by
        \begin{equation}
        \begin{aligned}
            &\minimize_{\bw,b,\bxi}& &\frac12 \|\bw\|_2^2 + C\sum_{i=1}^{l}\xi_i \\
            &\st& &\xi_i \geq 0, \quad i = 1,\dots,l, \\
            &&& y_i(\bw^{\T}\bx_i + b) \geq 1 - \xi_i, i = 1,\dots,l,
        \end{aligned}
        \label{eq:svm:soft-margin}
        \end{equation}
        where $C>0$ is a fixed parameter.
        \begin{enumerate}
        \item Show that \eqref{eq:svm:soft-margin} is equivalent\footnote{
                      Two optimization problems are called equivalent if from a solution of one,
                      a solution of the other is readily found, and vice versa.} to
            \begin{equation}
                \minimize_{\bw,b} \quad \frac12 \|\bw\|_2^2 + C\sum_{i=1}^{l} \ell(y_i,\bw^{\T}\bx_i+b),
            \label{eq:svm:hinge}
            \end{equation}
            where $\ell(\cdot,\cdot)$ is the hinge loss defined by $\ell(y,z) \defeq \max\{1-yz,0\}$.
        \begin{proof}
            \quad Let $\xi_i = \ell(y_i,\bw^{\star\T}\bx_i+b)$, As \begin{equation*}
                \ell(y_i,\bw^{\star\T}\bx_i+b) = \max(0,1-y_i(\bw^{\star\T}\bx_i,+b)) 
                \end{equation*}
                The $\xi_i \geq 0,\quad i = 1,\dots,l$.\\
            When \begin{equation*}
                1 - y_i(\bw^{\star\T}\bx_i + b) > 0
            \end{equation*}
            The $\xi_i = 1 - y_i(\bw^{\star\T}\bx_i+b)$, so that:
            \begin{equation*}
                y_i(\bw^{\star\T}+b) = 1 - \xi_i
            \tag{2.a.1}
            \end{equation*}
            When \begin{equation*}
                1 - y_i(\bw^{\star\T}\bx_i + b) \leq 0
            \end{equation*}
            The $\xi_i = 0$, so that:
            \begin{equation*}
                y_i(\bw^{\star\T}\bx_i + b) &\geq 1 - 0 \geq 1 - \xi_i
            \tag{2.a.2}
            \end{equation*}
            Combine the (2.a.1) and (2.a.2), it is easy to acquire that \begin{equation*}
                y_i(\bw^{\star\T}\bx_i+b) \geq 1 - \xi_i, \quad i = 1,\dots,l.
            \end{equation*}
            Taking all above into consideration, the \eqref{eq:svm:hinge} can be transformed as:
            \begin{equation*}
            \begin{aligned}
                &\minimize_{\bw,b,xi_i}& &\frac12 \|\bw\|_2^2+C\sum_{i=1}^{l}\xi_i \\
                &\st& &\xi_i \geq 0, \quad i = 1,\dots,l,\\
                &&& y_i(\bw^{\star\T}\bx_i+b) \geq 1 - \xi_i,\quad i = 1,\dots,l.
            \end{aligned}
            \end{equation*}
            Which is equivalent to the \eqref{eq:svm:soft-margin}.
        \end{proof}

        \item Show that the objective function of \eqref{eq:svm:hinge}, denoted by $f(\bw,b)$,
        is convex, i.e.,
            \begin{equation}
                f(\theta\bw_1+(1-\theta)\bw_2,\theta b_1 + (1-\theta)b_2) &\leq \theta f(\bw_1,b_1)
                + (1-\theta)f(\bw_2,b_2).
            \label{eq:svm:obj}
            \end{equation}
        for all $\bw_1,\bw_2 \in \reals^{n}, b_1, b_2 \in \reals$, and $\theta \in [0, 1]$.
        
        \begin{proof}
            The left part of the \eqref{eq:svm:obj}: 
            \begin{subequations} \label{eq:svm:obj_left}
                \begin{align}
                &f(\theta\bw_1+(1-\theta)\bw_2,\theta b1+(1-\theta)b2) \nonumber \\
                & =\frac{\theta^2}{2}\bw_1^{\T}\bw_1+
                  \frac{(1-\theta)^2}{2}\bw_2^{\T}\bw_2 + 
                  \frac{\theta(1-\theta)}{2}(\bw_1^{\T}\bw_2+\bw_2^{\T}\bw_1) \label{eq:svm:obj_left:1} \\
                & +C\sum_{i=1}^{l}\ell
                    \left(y_i,(\theta\bw_1+(1-\theta)\bw_2)^{\T}x_i+\theta b_1+(1-\theta)b_2\right)
                    \label{eq:svm:obj_left:2} 
            \end{align}
            \end{subequations}

            The right part of the \eqref{eq:svm:obj}:
            \begin{subequations}\label{eq:svm:obj_right}
                \begin{align}
                &\theta f(\bw_1,b_1)+(1-\theta)f(\bw_2,b_2) \nonumber \\
                &=\frac{\theta}{2}\bw_1^{\T}\bw_1+\frac{(1-\theta)}{2}\bw_2^{\T}\bw_2 
                    \label{eq:svm:obj_right:1} \\
                &+\theta C\sum_{i=1}^{l}\ell(y_i,\bw_1^{\T}x_i+b_1)+
                    (1-\theta)C\sum_{i=1}^{l}\ell(y_i,\bw_2^{\T}x_i+b_2) \label{eq:svm:obj_right:2}
                \end{align}
            \end{subequations}
            The \eqref{eq:svm:obj_right:1} - \eqref{eq:svm:obj_left:1} is:
            \begin{equation*}
            \begin{aligned}
                &\frac{\theta^2}{2}(\bw_1-\bw_2)^{\T}(\bw_1-\bw_2)
                +\frac{\theta}{2}(\bw_1^{\T}\bw_2+\bw_2^{\T}\bw_1-\bw_2^{\T}\bw_2)
                    +\frac{(1-\theta)}{2}\bw_2^{\T}\bw_2 &\\
                &-(\frac{\theta}{2}\bw_1^{\T}\bw_1 +
                    \frac{(1-\theta)}{2}\bw_2^{\T}\bw_2)& \\
                =& \frac{\theta}{2}\left(\bw_1^{\T}\bw_1-\theta(\bw_1^{\T}\bw_1+\bw_2^{\T}\bw_2-
                    \bw_1^{\T}\bw_2-\bw_2^{\T}\bw_1)+
                    \bw_2^{\T}\bw_2-\bw_1^{\T}\bw_2-\bw_2^{\T}\bw_1\right) &\\
                =& \frac{\theta(1-\theta)}{2}(\bw_1^{\T}\bw_1+\bw_2^{\T}\bw_2
                    -\bw_1^{\T}\bw_2-\bw_2^{\T}\bw_1) &\\
                =& \frac{\theta(1-\theta)}{2}(\bw_1-\bw_2)^{\T}(\bw_1-\bw_2) \geq 0.&
            \end{aligned}
            \end{equation*}
            So that the $ \eqref{eq:svm:obj_left:1} \leq \eqref{eq:svm:obj_right:1}$.\\

            Similarily, let $\bz_{1,i}=\bw_1^{\T}\bx_i+b_1$ and $\bz_{2,i}=\bw_2^{\T}\bx_i+b_2$,\\

            The \eqref{eq:svm:obj_right:2} - \eqref{eq:svm:obj_left:2} is:
            \begin{subequations}
            \begin{align}
                &\theta C\sum_{i=1}^{l}\ell(y_i,\bz_{1,i})+
                    (1-\theta)C\sum_{i=1}^{l}\ell(y_i,\bz_{2,i})
                -C\sum_{i=1}^{l}\ell(y_i,\theta\bz_{1,i}+(1-\theta)\bz_{2,i}) \nonumber &\\
                = & C\theta\sum_{i=1}^{l}\max\{0,1-y_i\bz_{1,i}\} 
                    +C(1-\theta)\sum_{i=1}^{l}\max\{0,1-y_i\bz_{2,i}\} \label{eq:7a}\\
                &- C\sum_{i=1}^{l}\max\{0,1-y_i(\theta\bz_{1,i}+(1-\theta)\bz_{2,i})\} \label{eq:7b}&
            \end{align}
            \end{subequations}
            Since $\max\{u,v\} = \frac12(u+v+|u-v|)$,
            \begin{equation*}
            \begin{aligned}
                \eqref{eq:7a} =& C\sum_{i=1}^{l}\left[1-y_i(\theta\bz_{1,i}+(1-\theta)\bz_{2,i})+
                    |1-y_i(\theta\bz_{1,i}+(1-\theta)\bz_{2,i})|\right]\\
                \eqref{eq:7b} =& C\sum_{i=1}^{l}\left[1-\theta y_i\bz_{1,i}-(1-\theta)y_i\bz_{2,i}+
                    |\theta(1-y_i\bz_{1,i})|+|(1-\theta)(1-y_i\bz_{2,i})|\right]
            \end{aligned}
            \end{equation*}
            The \eqref{eq:svm:obj_right:2} - \eqref{eq:svm:obj_left:2} is:
            \begin{equation*}
            \begin{aligned}
                \eqref{eq:7b}-\eqref{eq:7a} =& C\sum_{i=1}^{l}\left[|\theta(1-y_i\bz_{1,i})|+
                    |(1-\theta)(1-y_i\bz_{2,i})| \\
                    &- |\theta(1-y_i\bz_{1,i})+(1-\theta)(1-y_i\bz_{2,i})|\right] \geq 0
            \end{aligned}
            \end{equation*}
            That is the $\eqref{eq:svm:obj_left:2} \leq \eqref{eq:svm:obj_right:2}$.\\
            With all above, the $\eqref{eq:svm:obj_left:1}+\eqref{eq:svm:obj_left:2} \leq 
                \eqref{eq:svm:obj_right:1}+\eqref{eq:svm:obj_right:2}$, 
                so the \eqref{eq:svm:obj} is proved to be true.
        \end{proof}
        \end{enumerate}
\end{enumerate}
\end{document}
